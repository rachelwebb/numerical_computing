\lab{Appendix}{More Matplotlib}{More Matplotlib}
\label{mpltables}

\objective{Display relevant matplotlib information}

The documentation for matplotlib can be frustrating to maneuver and relevant information is sometimes difficult to find. For this reason, we have supplied some of the more applicable and useful information here. 
For a more general introduction to matplotlib, see lab \ref{lab:Matplotlib_and_Mayavi}.


\section*{Colours}
By default, every plot is assigned a different color specified by a ''color cycle". It can be overwritten by specifying what color is desired. \li{matplotlib} recognizes some basic built-in colours. 

\begin{tabular}
{|l||l|}
\hline
color & command \\
\hline
black & k \\
green & g \\
blue & b \\
red & r \\
cyan & c \\
magenta & m \\
yellow & y \\
white & w\\
\hline

\end{tabular}


The following displays how these colours can be  implemented. All the built-in colours are displayed for your convenience in Figure \ref{colours}
\lstinputlisting[style=fromfile]{colours.py}

\begin{figure} 
\includegraphics[width=\textwidth]{colours.pdf}
\caption{A display of all the built-in colours.}
\label{colours} 
\end{figure}


There are many other ways to specific colours. A popular method to access colours that are not built-in is to use a \li{RGB} tuple. Another way is to specify the color using an html hex string or its associated html color name like \li{DarkOliveGreen, FireBrick, LemonChiffon, MidnightBle, PapayaWhip,}, or \li{SeaGreen}. 

\section*{Axes}
You may have noticed the use of \li{plt.ylim([ymin, ymax])} in the previous code. This explicitly sets the boundary of the y-axis. Similarily, \li{plt.xlim([xmin, xmax])} can be used to set the boundary of the x-axis. 
Doing both commands simultaneously is possible with the \li{plt.axis([xmin, xmax, ymin, ymax])}. 
Remember that these commands must be executed after the plot. 

Another popular command is \li{plt.axis("tight")} which adjusts the axes so all data is visible and centered. 

\section*{lines}
\subsection*{width}
You may have noticed that the width of the lines above seemed thin considering we wanted to inspect the line colour. \li{linewidth} is a keyword argument that is defaulted to be \li{None} but can be given any real number to adjust the line width. 

The following displays how \li{linewidth} is implemented. It is displayed in Figure \ref{linewidth}
\lstinputlisting[style=fromfile]{linewidth.py}

\begin{figure} 
\includegraphics[width=\textwidth]{linewidth.pdf}
\caption{plot of varying linewidths.}
\label{linewidth} 
\end{figure}

\subsection*{style}
By default, plots are executed with a solid line style. We can easily change that. The following are accepted format string characters to indicate line style. 

\begin{tabular}
{|c||l|}
\hline
character & description \\
\hline
- & solid line style \\
-{}- & dashed line style\\
-. & dash-dot line style \\
: & dotted line style \\
. & point marker \\
, & pixel marker \\
o & circle marker \\
v & triangle\_down marker \\
\^{} & triangle\_up marker \\
$<$ & triangle\_left marker \\
$>$ & triangle\_right marker \\
1 & tri\_down marker \\
2 & tri\_up marker \\
3 & tri\_left marker \\
4 & tri\_right marker \\
s & square marker \\
p & pentagon marker \\
* & star marker \\
h & hexagon1 marker \\
H & hexagon2 marker \\
+ & plus marker \\
x & x marker \\
D & diamond marker \\
d & thin\_diamond marker \\
$|$ & vline marker \\
\_{} & hline marker \\
\hline 
\end{tabular}


The following displays how \li{linestyle} can be  implemented. 

It is displayed in Figure \ref{linestyle}
\lstinputlisting[style=fromfile]{linestyle.py}

\begin{figure} 
\includegraphics[width=\textwidth]{linestyle.pdf}
\caption{plot of varying linestyles.}
\label{linestyle} 
\end{figure}


\section*{Text}
It is also possible to add text to your plots. 

To label your axes, the \li{plt.xlabel()} and the \li{plt.ylabel()} can both be used. 

The function \li{plt.title()} will add a title to a plot.
If you are working with subplots, this command will add a title to the subplot you are currently modifying.
To add a title above the entire figure, use \li{plt.suptitle()}.


All of the \li{text()} commands can be customized with \li{fontsize} and \li{color} keyword arguments. 

We can add these elements to our previous example. 

It is displayed in Figure \ref{text}
\lstinputlisting[style=fromfile]{text.py}

\begin{figure} 
\includegraphics[width=\textwidth]{text.pdf}
\caption{plot of varying linestyles using text labels.}
\label{text} 
\end{figure}

